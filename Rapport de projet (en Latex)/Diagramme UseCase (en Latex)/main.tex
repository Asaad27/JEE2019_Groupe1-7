\documentclass[14pt]{article}
\usepackage{geometry} % see geometry.pdf on how to lay out the page. There's lots.
\usepackage{tikz}
\usetikzlibrary{%
arrows,
shapes,
chains,
matrix,
positioning,
shapes.multipart,
calc,
scopes}
\usepackage{tikz-uml} 
\usepackage{ifthen}
\usepackage{xstring}
\usepackage{pgfkeys}


\begin{document}
\begin{center}
\begin{tikzpicture}
\begin{umlsystem}[x=4, fill=red!10]{E-boutique de livre} 
  \umlusecase[name=case1,x=3]{consulter livres}
  \umlusecase[name=case13,x=3,y=-1]{creer un compte} 
  \umlusecase[name=case2,x=3,y=-2.7, width=1.7cm]{ajouter des livres au panier}
  \umlusecase[name=case3,x=3,y=-4.7, fill=green!20]{s'authentifier}
  \umlusecase[name=case7,y=-7]{enregistrer panier}
  \umlusecase[name=case8,y=-9, width=2cm]{passer commande}
  \umlusecase[name=case9,y=-11, width=2cm]{consulter ses commandes}
  \umlusecase[name=case6,x=6, y=-7]{gerer les categories}
  \umlusecase[name=case10,x=6, y=-8.5]{gerer les livres}
  \umlusecase[name=case11,x=6, y=-10]{gerer les clients}
  \umlusecase[name=case12,x=6, y=-11.5]{gerer les commandes}

\end{umlsystem}

\umlactor{visiteur} 
\umlactor[y=-6.5]{client} 
\umlactor[x=14.5, y=-6.5]{administrateur}

\umlinherit{client}{visiteur}
\umlassoc{visiteur}{case1}
\umlassoc{visiteur}{case2} 
\umlassoc{visiteur}{case13}
\umlassoc{client}{case3}
\umlassoc{client}{case7}
\umlassoc{client}{case8} 
\umlassoc{client}{case9} 
\umlassoc{administrateur}{case3} 
\umlassoc{administrateur}{case6}
\umlassoc{administrateur}{case10}
\umlassoc{administrateur}{case11}
\umlassoc{administrateur}{case12}

\umlinclude[name=incl]{case7}{case3}
\umlinclude[name=incl]{case8}{case3}
\umlinclude[name=incl]{case9}{case3}
\umlinclude[name=incl]{case6}{case3}
\umlinclude[name=incl]{case10}{case3}
\umlinclude[name=incl]{case11}{case3}
\umlinclude[name=incl]{case12}{case3}

\end{tikzpicture}
\end{center}
\end{document}
